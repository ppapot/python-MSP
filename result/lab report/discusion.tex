\section*{Discussion}
The different graph (graph (a) and (b)) presented in the result section show that the number of postive and negative pions as function of pseudorapidity and transvers momentum are quite similar. This resemblance could initially suggest a symmetry between them. 
\\

However, the difference graphs (graph (c) and (d)) show that the differences do not equal zero in certain bins, indicating potential asymmetry in those ranges. 

Moreover, the trend observed in graph(d), of increasing differences in lower pt ranges(0 to 1.5 GeV/c) compared to higher ranges suggest that PYTHIA may treat the creation of particles and antiparticles differently as a function of energy. In this context, lower energy appears to correlate with a higher production rate of pions relative to anti-pions. Similarly for pseudorapidity, the observed trend in graph (c) shows increasing differences in lower $\eta$ ranges (-4 to -1) as well as in higher ranges (1 to 4). This might imply that the mechanisms PYTHIA use for governing their pion production can be influenced by the angle of emission relative to the collision axis. This suggests that certain angles, particularly those further from the beam axis, may favor the production of pions over anti-pions.
\\

However before drawing conclusions, the significance of the results need to be evaluated. This involves determining how many standard deviation ($\sigma$) the values are from the reference point, which is a mean difference of zero representing the null hypothesis of no asymmetry. The number of $\sigma$ can be found by comparing the mean difference to its associated uncertainty using the following formula:

\[\frac{\text{mean difference}}{\text{uncertainty}}\approx n \sigma\]

Applying this formula to the mean differences per bin of pseudorapidity and transverse momentum (as shown in graph (c) and (d)) yield $\sigma$ ranges from 0.48$\sigma$ to  2.61$\sigma$ for pseudorapidity and from 0.29$\sigma$ to  1.92$\sigma$ for transverse momentum. None of the values are above the 3$\sigma$ threshold (indicating statistical significance at a 99.7\% confidence level, the common minimum required to draw any conclusion), meaning that the results per bins are not statistically significant. This weakens the suggested asymmetry, as the observed differences could be due to random fluctuations. It is important to note that non-significant results don't prove symmetry; they indicate that no definitive conclusion can be made from the current data. Further investigation are needed to assess the effects of pseudorapidity and transverse momentum on pions anti-pions production.    
\\

While the pseudorapidity and transvers momentum results fail to show significance, the analysis of the pion counts present a different outcome. Similarly, the significance of the mean difference between the number of positive and negative pions (as shown in table 1) can be determined. Using the previous formula, the number of sigma found for this result is $\approx$ 7.5$\sigma$. This means the value is highly significant, as it largely exceed the 3$\sigma$ threshold. Even though the difference is really small, the high significance suggest that indeed there is more matter than anti matter following the collision of protons. Thereby this result support the hypothesis of this paper.
\\

In conclusion, the analysis indicates that the overall averages of pion and anti-pion production support the hypothesis of matter-antimatter asymmetry following proton collisions, implying that PYTHIA captures fundamental trends, such as a slight excess of matter over antimatter. However, the lack of statistical significance in the differences based on pseudorapidity and transverse momentum suggests that these factors may not be responsible for the observed asymmetry, or that PYTHIA may oversimplify these aspects of particle interactions. Consequently, future experiments are necessary to further investigate this topic, potentially focusing on different collision energies, particle types, or refining the modeling approaches to better understand the dynamics of matter-antimatter production.