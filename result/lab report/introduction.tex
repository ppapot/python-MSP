\section*{Introduction}

In the quest of understanding the origin of universe, scientists elaborated the Big Bang Theory. This theory suggests that at the universe's inception, both matter and antimatter were created. However, we observe that, in our current universe, everything is made of matter. So, where did antimatter go? (\cite{robson2018}) 
\\

To address this question, we must first understand what antimatter is and how it interacts with matter. Antimatter is considered the mirror image of matter, while both share the same properties, like mass, they have opposite charges. For each particle in the universe, there exists its corresponding antiparticle: electrons have positrons as their corresponding antiparticle, in the same way that protons have instead antiprotons. When matter and antimatter interact, they annihilate each other (\cite{cern_matter_antimatter_asymmetry}).
\\

Scientists suspect there was an asymmetry in the amount of matter and antimatter in the early stage of our universe. After most of the matter and antimatter annihilated each other, a small excess of matter remained. This imbalance caused matter to prevail over antimatter, leading to the creation of all the things that surrounds us today and the absence of antimatter (\cite{cern_matter_antimatter_asymmetry}). The existence of this asymmetry is what researchers are still trying to prove.
\\

Recent technologies, such as particle accelerator, offer new opportunities to explore this question. Particle accelerators are used to facilitate collision between particles (like protons). These protons are accelerated in opposite directions in a large circular structure, thanks to electrical fields. They can reach extremely high speed (near the speed of light), allowing an enormous release of energy when they collide. This energy is then available to produce new particle of matter and antimatter (\cite{DOE2024}).
\\  

These accelerators allow scientists to study particles that are not commonly found, like pions, which are the focus of this paper.
Pions, also known as pi mesons, are composed of quarks and anti-quarks and are among the lightest mesons. Positive pions are made of an up quark and an anti-down quark, while negative pions are made of a down quark and an anti-up quark. They are antiparticles of each other. Pions are known for their role in mediating the strong nuclear force, which binds protons and neutrons together in atomic nuclei. (\cite{pasayten2021})
\\

This paper will investigate whether there is an asymmetry of matter and antimatter produced following protons collision. To do so, we will analyze data samples from a model, PYTHIA, which simulates these kinds of collisions, as we unfortunately did not have access to a particle accelerator. As mentioned, we will focus on pions. More specifically, we will investigate momentum of positive and negative pions. 


